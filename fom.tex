% allgemeine Einstellungen
\documentclass[12pt,a4paper]{report}
\usepackage[T1]{fontenc}
\usepackage[utf8]{inputenc}
\usepackage[ngerman]{babel}
\usepackage{txfonts}
\usepackage{graphicx}

\usepackage[dvipsnames]{xcolor}

%% geometry
% Seitenränder einstellen
% ggf. includefot wieder entfernen
\usepackage{geometry}
\geometry{a4paper, top=3.0cm, left=4.0cm, right=2.0cm, bottom=2.0cm, includefoot}

%% biblatex
% Quellenverzeichnis für deutsche Sprache anpassen
% Zitats und Autorenstil anpassen
\usepackage[backend=bibtex8, citestyle=authoryear, bibstyle=authoryear]{biblatex}
\usepackage[babel,german=guillemets]{csquotes}
% Quellen-Datei "fom.bib" einlesen
\bibliography{fom}


%% Einrückung
% Keine Einrückung bei neuen Zeilen
\setlength{\parindent}{0pt}

%% fancyhdr
% verwendet um header und footer zu überarbeiten\normalsize
\usepackage{fancyhdr}
\pagestyle{fancy}
% header überschreiben und oben rechts die aktuelle seite zeigen
\fancyhead{}
\fancyhead[RO]{\thepage}
% footer überschreiben
\fancyfoot{}
% Keine Trennlinie unter dem Header anzeigen
\renewcommand{\headrulewidth}{0.0pt}
% Auf jeder Seite Fancy-Stile aktivieren (verhindert Überschreiben von Inhaltsverzeichnis etc)
\renewcommand{\bibsetup}{\thispagestyle{fancy}}

%% sectsty
% Schrift und Zählung von Kapitel
\usepackage{sectsty}
\sectionfont{\fontsize{12}{15}\selectfont}
\subsectionfont{\fontsize{12}{15}\selectfont}
\subsubsectionfont{\fontsize{12}{15}\selectfont}
\setcounter{secnumdepth}{4}
\setcounter{tocdepth}{4}

%% setspace
% 1,5 Zeilenabstand
\usepackage{setspace}
\onehalfspacing

%% tocstyle
% Unterpunkte rechtsbündig im Inhaltsverzeichnis
\usepackage[tocfullflat]{tocstyle}

%% Quell-Code
\usepackage{times}
\usepackage{listings}
\lstset{
    language=Ruby,
    tabsize=2,   
    basicstyle=\ttfamily\footnotesize,
    breaklines=true,
    prebreak=\raisebox{0ex}[0ex][0ex]{\ensuremath{\hookleftarrow}},
    frame=lines,
    showtabs=false,
    showspaces=false,
    showstringspaces=false,
    keywordstyle=\color{blue}\bfseries,
    stringstyle=\color{green!50!black},
    commentstyle=\color{gray}\itshape,
    numbers=left,
    numbersep=5pt,                   % how far the line-numbers are from the code
    numberstyle=\tiny\color{gray},    
    captionpos=b,
    escapeinside={\%*}{*)}
}

%% uml
\usepackage{tikz}
\usepackage{pgf-umlsd}
\usepgflibrary{arrows} % for pgf-umlsd
\usepackage{verbatim}

\begin{document}
% Titel bauen
\noindent
\textbf{FOM Hochschule für Oekonomie \& Management Essen}\\
\textbf{Standort Stuttgart}\\
\linebreak
\linebreak
\textbf{Berufsbegleitender Studiengang zum}\\
\textbf{Master of IT-Management}\\
\linebreak
\linebreak
\textbf{2. Semester}\\
\linebreak
\linebreak
\linebreak
\textbf{Seminararbeit in IT-Projektmanagement und Software-Engineering}\\
\linebreak
\linebreak
\textbf{Realisierung eines Model-View-Controller-Architekturmodells am Beispiel\linebreak des Frameworks Ruby on Rails}\\
\linebreak
\linebreak
Betreuer/in: Prof. Dr. Cornelia Heinisch\\
Autor: Kai-Markus Lüer\\
Matrikelnr.: 340888\\
Abgabedatum: \today
\thispagestyle{empty}

\setcounter{page}{1}
\clearpage

% ab hier römische Seit\fontsize{...}{...} enzahlen
\renewcommand{\thepage}{\roman{page}}

\section*{Kurzfassung}
% Kapitel direkt dem Inhaltsverzeichnis zufügen
\addcontentsline{toc}{section}{Kurzfassung}
Diese Arbeit stellt die Realisierung des Model-View-Controller-Architekturmodells (MVC) am Beispiel einer Ruby-on-Rails-Anwendung dar. Nachdem die benötigten Grundbegriffe definiert wurden, wird das MVC-Architekturmodell vorgestellt, folgend das Framework Ruby on Rails. Anhand einer Beispielanwendung werden die im Framework verwendeten Techniken gezeigt, die das MVC-Architekturmodell realisieren. Abschließend wird im Fazit bewertet, wie das Framework das MVC-Architekturmodell realisiert hat.
% anzeige vom Kapitel, wird aber nicht gezähl
\newpage

% Inhaltsverzeichnis
\addtocontents{toc}{\protect\thispagestyle{fancy}}
\addcontentsline{toc}{section}{Inhaltsverzeichnis}
\renewcommand\contentsname{\fontsize{12}{15}\selectfont Inhaltsverzeichnis}
\tableofcontents
\newpage



	
\section*{Abkürzungsverzeichnis}
\thispagestyle{fancy}
\addcontentsline{toc}{section}{Abkürzungsverzeichnis}
	
\addcontentsline{toc}{section}{Abbildungsverzeichnis}
\renewcommand{\listfigurename}{\fontsize{12}{15}\selectfont Abbildungsverzeichnis}
\thispagestyle{fancy}
\listoffigures
\thispagestyle{fancy}
\addcontentsline{toc}{section}{Listingverzeichnis}
\renewcommand{\lstlistlistingname}{\fontsize{12}{15}\selectfont Listingverzeichnis}
\lstlistoflistings
\thispagestyle{fancy}

\newpage

% Ab hier arabische Seitenzahlen
\renewcommand{\thepage}{\arabic{page}}
% Ab hier arabische Kapitelzählung
\renewcommand{\thesection}{\arabic{section}}

% Kapitel bei 1 anfangen lassen
\setcounter{section}{0}

% Seite bei 1 anfangen lassen
\setcounter{page}{1}
\include{./kapitel/einleitung}
\include{./kapitel/hintergrund}
\include{./kapitel/mvc}
\include{./kapitel/rails}
\include{./kapitel/mvc_in_rails}
\include{./kapitel/fazit}
\include{chapter_2}


%% Quellenverzeichnis
% Im Inhaltsverzeichnis eintragen
\begingroup
	\let\clearpage\relax
	\section{Literatur- und Quellenverzeichnis}
	\printbibliography[title={\fontsize{12}{15}\selectfont Literatur-Quellen}, type=book]
	\printbibliography[title={\fontsize{12}{15}\selectfont Online-Quellen}, type=online]
\endgroup


\end{document}