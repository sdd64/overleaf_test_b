\documentclass[11pt,twocolumn,fleqn]{article}

\usepackage[english]{babel}
\usepackage[utf8x]{inputenc}
\usepackage{amsmath}
\usepackage{graphicx}
\usepackage[colorinlistoftodos]{todonotes}
\usepackage[top=0.3in, bottom=0.3in, left=0.3in, right=0.5in]{geometry}
\usepackage{titlesec}
\usepackage{tikz}
\usetikzlibrary{trees}
\titleformat*{\section}{\footnotesize\bfseries}
\titleformat*{\subsection}{\footnotesize\bfseries}
\titleformat*{\subsubsection}{\footnotesize\bfseries}
\titleformat*{\paragraph}{\large\bfseries}
\titleformat*{\subparagraph}{\large\bfseries}
%% Einrückung
% Keine Einrückung bei neuen Zeilen
\setlength{\parindent}{0pt}


\begin{document}
\footnotesize

\section{Kapitel 3}

\textbf{Arithmetische Mittel}
\begin{equation*}
\overline{x} = \frac{1}{n}(x_1+x_2+...+x_n)=\frac{1}{n}\sum_{i=1}^{n}x_i
\end{equation*}
\begin{center}\begin{tabular}{rcl}
   $x_i$ & = & i-te Merkmalsausprägung  \\
   $n$ & = & Anzahl der Merkmalsausprägungen \\
\end{tabular}\end{center}


\textbf{Ganzzahlig gewichtetes Mittel}
\begin{equation*}
\overline{x}_g = \frac{\sum_{i=1}^{n}g_ix_i}{\sum^{n}_{i=1}g_i}
\end{equation*}

\textbf{Modus}
\begin{equation*}
Modus = x_u + \frac{h_0-h_{0p}}{2* h_0-h_{0p}-h_{0s}} * w
\end{equation*}
\begin{center}\begin{tabular}{rcl}
   $x_u$ & = & Klassenuntergrenze der Klasse, in die der Modus fällt  \\
   $h_0$ & = & Häufigkeit dieser Klasse \\
   $h_{0p}$ & = & Häufigkeit der vorgehenden Klasse von $h_0$  \\
   $h_{0s}$ & = & Häufigkeit der nachfolgenden Klasse von $h_0$ \\
   $w$   & = &  Gemeinsame Klassenbreite \\
\end{tabular}\end{center}


\textbf{Median}
\begin{equation*}
Median = x_u + \frac{ \frac{n+1}{2}-h_u }{h_e} *w
\end{equation*}
\begin{center}\begin{tabular}{rcl}
   $x_u$ & = & Klassenuntergrenze der Klasse, in die der Median fällt  \\
   $h_u$ & = & Häufigkeit aller vorgehenden Klasse \\
   $h_e$ & = & Häufigkeit der Median-Klasse \\
   $n$   & = & Anzahl der Elemente \\
   $w$   & = &  Breite der Median-Klasse \\
\end{tabular}\end{center}


\textbf{Geometrisches Mittel}
\begin{equation*}
geomatrisches Mittel = \sqrt[n]{x_i*x_2*...x_n}
\end{equation*}
\begin{center}\begin{tabular}{rcl}
   $x_u$ & = & i-te Merkmalsausprägung  \\
   $n$   & = & Anzahl der Merkmalsausprägungen \\
\end{tabular}\end{center}

\textbf{Streuungsmaß}
\begin{equation*}
\overline{s}=\frac{1}{n}\sum^n_{i=1}|x_1-\overline{x}|
\end{equation*}
\begin{center}\begin{tabular}{rcl}
   $x_u$ & = & i-te Merkmalsausprägung  \\
   $n$   & = & Anzahl der Merkmalsausprägungen \\
\end{tabular}\end{center}

\textbf{Streuungsmaß bei Klassen}
\begin{equation*}
\overline{s}=\frac{1}{n}\sum^m_{k=1}h_k|x_k-\overline{x}|
\end{equation*}
\begin{center}\begin{tabular}{rcl}
   $h_k$ & = & Häufigkeit der k-ten Klasse  \\
   $x_k$ & = & Klassenmitte der k-ten Klasse \\
   $\overline{x}$ & = & Mittelwert \\
   $m$   & = & Anzahl der Klassen \\
   $n$   & = &  Anzahl der Merkmalsausprägungen \\
\end{tabular}\end{center}

\textbf{Varianz (Grundgesamtheit)}
\begin{equation*}
\sigma^2=\frac{1}{n}\sum^n_{i=1}(x_i-\overline{x})^2
\end{equation*}

\textbf{Varianz (Stichprobe)}
\begin{equation*}
\sigma^2=\frac{1}{n-1}\sum^n_{i=1}(x_i-\overline{x})^2
\end{equation*}

\textbf{Variationskoeffizient}
\begin{equation*}
Variationskoeffizient = \frac{\sigma * 100}{\overline{x} } 
\end{equation*}


\section{Kapitel 4}

\textbf{Wahrscheinlichkeit}
\begin{equation*}
P(E_i)=\frac{H(E_i)}{n}
\end{equation*}

\textbf{Statistische Wahrscheinlichkeit}
\begin{equation*}
P(E_i)= \lim_{n \to \infty} \frac{H(E_i)}{n}
\end{equation*}

\textbf{Wahrscheinlichkeitsfunktion}
\begin{equation*}
f(x)=P(X=x)= \begin{cases} 
p_j & \text{für}  x = x_j \\ 
0 & sonst  \\ 
\end{cases}
\end{equation*}

\textbf{Verteilungsfunktion}
\begin{equation*}
F(x)=P(X \leq x)=\sum_{x_j \leq x }f(x_j)
\end{equation*}

\textbf{Berechnen einer Reihe von Wahrscheinlichkeiten}
\begin{equation*}
P(a<X \leq b) = F(b)-F(a) \sum_{x_j \leq b }f(x_j) - \sum_{x_j \leq a }f(x_j)
\end{equation*}

\textbf{Stetige Wahrscheinlichkeit}
\begin{equation*}
F(x) = P(X \leq x) = \int^x_{-\infty} f(v)dv
\end{equation*}

\textbf{Bedingte Wahrscheinlichkeit $P=(B|A)$}
\begin{equation*}
P(A \cap B) = P(A) * P(B)\\
P(A \cup B) = P(A) + P(B)
\end{equation*}

Satz von Bayes (allgemein)
\begin{equation*}
P(A|B) = \frac{ P(B|A) * P(A) }{ P(B) }
\end{equation*}

\textbf{Der Patient ist nach postiven Test krank}
\begin{equation*}
P(krank|positiv) = \frac{H(positiv | krank)}{H(positiv|krank) + H(positiv|gesund)}
\end{equation*}

\textbf{Tip: Summe von Wahrscheinlichkeiten}
\begin{equation*}
P(\neg A) = 1 - P(A)
\end{equation*}

\textbf{Wahrscheinlichkeitsbaum}
% Set the overall layout of the tree
\tikzstyle{level 1}=[level distance=2.5cm, sibling distance=2.5cm]
\tikzstyle{level 2}=[level distance=2.0cm, sibling distance=2.0cm]

% Define styles for bags and leafs
\tikzstyle{bag} = [text width=4em, text centered]
\tikzstyle{end} = [circle, minimum width=3pt,fill, inner sep=0pt]
\begin{tikzpicture}[grow=right, sloped]
\node[bag] {Produkte $100$}
    child {
        node[bag] {$20$}        
            child {
                node[end, label=right:
                    {$P(S2 \cap def.)= 0,2 * 0,05 $}] {}
                edge from parent
                node[above] {defekt}
                node[below]  {$5\%$}
            }
            child {
                node[end, label=right:
                    {$P(S1 \cap gut)= 0,2 * 0,95 $}] {}
                edge from parent
                node[above] {gut}
                node[below]  {$95\%$}
            }
            edge from parent 
            node[above] {$S2$}
            node[below]  {prod. 20}
    }
    child {
        node[bag] {80}        
        child {
                node[end, label=right:
                    {$P(S1 \cap def.)= 0,8 * 0,1$}] {}
                edge from parent
                node[above] {defekt}
                node[below]  {$10\%$}
            }
            child {
                node[end, label=right:
                    {$P(S1 \cap gut)=0,8 * 0,9$}] {}
                edge from parent
                node[above] {gut}
                node[below]  {$90\%$}
            }
        edge from parent         
            node[above] {$S1$}
            node[below]  {prod. 80}
    };
\end{tikzpicture}

Wie groß sind die Wahrscheinlichkeiten, dass ein Produkt am Standort S2 produziert worden ist, wenn das Produkt defekt ist?
$P(produziert S2 | defekt) = (0,2 * 0,05) / (0,2*0,05 + 0,1 * 0,8) = 0,11$ 


\section{Kapitel 6}

\textbf{Schätzwert eines Stichprobenumfangs n}
\begin{equation*}
n \geq p (\frac{q}{e})^2
\end{equation*}

\begin{equation*}
p = \begin{cases} 
\sigma^2 & \text{bei quantitativen Merkmalen} \\ 
P*(1-P) & \text{bei qualitativen Merkmalen}  \\ 
\end{cases}
\end{equation*}

\begin{equation*}
e = \begin{cases} 
E & \text{bei quantitativen Merkmalen} \\ 
E*0.01 & \text{bei qualitativen Merkmalen}  \\ 
\end{cases}
\end{equation*}

E= z.B. ''5\% Intervallschätzung'', P="Abweichung von 20 Euro", oder P = Bekanntheitsgrad, z.B. ''75\%''. Falls kein Bekanntheitsgrad angegeben P=0.25  

\section{Kapitel 7}
\textbf{gleitender Durchschnitt 3. Ordnung}
\begin{equation*}
\overline{y}_t = \frac{y_{t-1} + y_t + y_{t+1}}{3}
\end{equation*}

\textbf{gleitender Durchschnitt 4. Ordnung}
\begin{equation*}
\overline{y}_t = \frac{ \frac{1}{2}y_{t-2} + y_{t-1} + y_t + y_{t+1} + \frac{1}{2}y_{t+2}}{4}
\end{equation*}

\textbf{gleitender Durchschnitt 5. Ordnung}
\begin{equation*}
\overline{y}_t = \frac{ y_{t-2} + y_{t-1} + y_t + y_{t+1} + y_{t+2}}{5}
\end{equation*}

\section{Kapitel 8}
\textbf{Korrelationskoeffizient}
\begin{equation*}
r(x,y)= \frac{s(x,y)}{\sigma(x)*\sigma(y)}
\end{equation*}

\textbf{Korrelationskoeffizient(Komplex, Stichprobe)}

  \begin{equation*}
  r(x,y)= \frac{s(x,y)}{\sigma(x)*\sigma(y)} = \frac{ \frac{1}{n-1}\sum^n_{i=1}(x_i-\overline{x})*(y_i-\overline{y}) }{ \sqrt[]{ \frac{1}{n-1}\sum^n_{i=1}(x_i - \overline{x})^2 } * \sqrt[]{ \frac{1}{n-1}\sum^n_{i=1}(y_i - \overline{y})^2 } }
  \end{equation*}
  
\textbf{Korrelationskoeffizient(Komplex, Grundgesamtheit)}

  \begin{equation*}
  r(x,y)= \frac{s(x,y)}{\sigma(x)*\sigma(y)} = \frac{ \frac{1}{n}\sum^n_{i=1}(x_i-\overline{x})*(y_i-\overline{y}) }{ \sqrt[]{ \frac{1}{n}\sum^n_{i=1}(x_i - \overline{x})^2 } * \sqrt[]{ \frac{1}{n}\sum^n_{i=1}(y_i - \overline{y})^2 } }
  \end{equation*}  

\textbf{Lineare Abhängigkeit}

Ist die Variable y von der Variablen x statistisch linear abhängig, d. h. es besteht eine Beziehung der Form y = a + b x, sind die Parameter der Regressionsgeraden wie folgt definiert: 

\begin{equation*}
b:=\frac{s(x,y)}{s(x)^2}\\
a:=\overline{y}-b*\overline{x}
\end{equation*}

\textbf{Bestimmtheitsmaß (komplex)}
Bestimmtheitsmaß B gibt an, wie gut die Beziehung der Größen durch die Regressionsgerade erklärt wird
\begin{equation*}
B:=\frac{ \sum^n_{i=1}(\widehat{y}_i-\overline{y})^2 }{ \sum^n_{i=1}(y_i-\overline{y})^2 }
\end{equation*}
\begin{center}\begin{tabular}{rcl}
   $\widehat{y}_i$ & = & Funktionswert auf der Regressionsgerade \\
   $y_i$ & = & Beobachtungswert \\
   $\overline{y}$ & = & Arithmetisches Mttel \\
\end{tabular}\end{center}

\textbf{Bestimmtheitsmaß (simpel)}
\begin{equation*}
B=(r(x,y))^2
\end{equation*}

\textbf{Regressionsanalyse}
\begin{center}
    \begin{tabular}{ l | l | l | l }
Beobachtung        & 1    & 2    & 3\\
x Preis pro Kugel  & 1,00 & 1,50 & 2\\ \hline
y Absatz pro Kugel & 5    & 5    & 2
    \end{tabular}
\end{center}

\begin{center}
    \begin{tabular}{ l | l | l | l | l | l }
i & $x_i$ & $y_i$ & $ (x_i - \overline{x})^2 $ & $(y_i - \overline{y})^2$ & $(x_i - \overline{x})(y_i - \overline{y})$ \\ \hline
1       & 1,0 & 5  & 0,25 & 1  & -0,5 \\ 
2       & 1,5 & 5  & 0    & 1  & 0    \\
3       & 2,0 & 2  & 0,25 & 4  & -1   \\ \hline
$\sum$  & 4,5 & 12 & 0,5  & 6  & -1,5 
    \end{tabular}
\end{center}
$ \overline{x}=1,5$ ; $\overline{y}=4$ \newline
$ s(x,y) = \frac{1}{3}*(-1,5) = -0,5 $ \newline
\begin{equation*}
  r(x,y) = \frac{-0,5}{ \sqrt[]{\frac{1}{3}(0,5) } * \sqrt[]{ \frac{1}{3}(6) } } = -0,87; B=(r(x,y)^2) = 0,7569 = 76\% 
\end{equation*}

\begin{equation*}
b = \frac{-0,5}{ \frac{1}{3}(0,5) } = -3; a = 6 - (-3 * 0,5) = 7,5; y = 7,5 - 3x 
\end{equation*}



\section{Übung 5: Korrelationsanalyse}
\textbf{Erwartungswert}
\begin{equation*}
\mu = \sum^n_{i=i}p_i*e_i
\end{equation*}

\textbf{Erwartungsnutzen}
\begin{equation*}
Y=EU(a_i)=\sum_kp_k*u(e_i,_k)
\end{equation*}


\textbf{Einzahlungen und Erwartungsnutzen}
B=400

\begin{center}
    \begin{tabular}{ l| l | l | l | l }
           & s1  & s2  & s3  & s4  \\ \hline
Ergebnis 1 & 40  & 80  & 40  & 80  \\
Ergebnis 2 & 100 & 120 & 120 & 100 \\ \hline
Warsch. \% & 20  & 30  & 10  & 40
    \end{tabular}
\end{center}

\begin{center}
    \begin{tabular}{ l | l | l | l | l | l l}
                   & s1  & s2  & s3  & s4 & \multicolumn{2}{c}{$ \sqrt[]{(e-min)/(max-min)} $}  \\
                   & 0.2 & 0.3 & 0.1 & 0.4 & Rendite[\%]$\mu$ & Risiko[\%]$\sigma$ \\ \hline
    a3(6x50,1x100) & 340 & 600 & 360 & 580 & \multicolumn{2}{c}{$EU(a)$ = 0.73} \\ \hline                   
    a3(6x50,1x100) & -15 & 50 & -10 & 45 & 29.0  & 27.8 \\ \hline
    \end{tabular}
\end{center}
$ 340 = (6*40)+(1*100); 600 = (6*80)+(1*120); ...  $

\begin{equation*}
e_{3,1} = \frac{b_{3,1}-b_0} {b_0}* 100 = \frac{340-400}{400}*100=-15
\end{equation*}

\begin{equation*}
e_{3,2} = \frac{b_{3,1}-b_0} {b_0}* 100 = \frac{600-400}{400}*100=50
\end{equation*}

\textbf{Korrelation der Anlagen 1 und 2}
\begin{equation*}
r(a_1,a_2)=\frac{ \sum^4_{k=1}(e_{1,k}-\mu_1)*(e_{2,k}-\mu_2)*p_k }{ \sigma_1 * \sigma_2 } = 0.09
\end{equation*}

\textbf{Erwartete Rendite}
\begin{equation*}
\mu(a_3) = x_1 * \mu_1 + x_2 * \mu_2 = (0,75*36)+(0,25*8)=29
\end{equation*}

\textbf{Erwartete Rendite(simpel)}
\begin{equation*}
\mu(a_3) = ((-15 * 0,2) + (50 * 0,3) + (-10 * 0,1) + (45 * 04)) / 4 = 29 
\end{equation*}

\textbf{Risiko der Rendite}
\begin{equation*}
\sigma(a_3) = \sqrt[]{x^2_1 * \sigma^2_1 + x^2_2 * \sigma^2_2 + 2 * x_1 * \sigma_1 * x_2 \sigma_2 * r (a_1,a_2) } = 27,8
\end{equation*}

\textbf{Risiko der Rendite(simpel)}
\begin{equation*}
\sigma(X)= \sqrt[]{\sum p(x)*(x - \mu(X))^2}
\end{equation*}
\begin{equation*}
\sqrt[]{0,2*(-15-29)^2 + 0,3 * (50-29)^2 + 0,1 * (-10-29)^2 + 0,4 * (45-29)^2 }
\end{equation*}

\textbf{Diversifikationseffekt (simpel)}
\begin{equation*}
\delta(a_3)=\sigma_{a3}-\mu_{a3} = 27.8 - 29.9 = -2.1
\end{equation*}

\end{document}
