\section{Einführung}
\subsection{Begriffe}
\begin{itemize}
  \item Grundgesamtheit: Menge aller potenziellen Untersuchungsobjekte, über die man eine statistische Aussage machen möchte
  \item Stichprobe: beschränkte Auswahl aus der Grundgesamtheit
  \item Repräsentativ: von einer Stichprobe kann man ggf. auch auf ein größeres Ganzes (also die Grundgesamtheit) schließen
  \item Teilerhebung: Abwandlung von Stichprobe, zb. ''jeder 2. Student''
  \item Vollerhebung: Jedes Mitglied der Grundgesamtheit
  \item Verzerrung: Bias, systematischer Fehler im Datenauswahlverfahren oder der Daten selber. Liegt oftmals an nicht beachteten Störgrößen. Auch Beeinflussung beim Erheben.
  \item Einzelobjekt: einzelnene Einheit aus der Stichprobe / Grundgesamtheit. Auch als \item Merkmalsträger bezeichnet. Hat mehrere Merkmale. (z.B. Erna Müller)
  \item Merkmal: Interessierende Eigenschaften, über die man Informationen haben möchte (z.B. Alter)
  \item Merkmalsausprägungen: Mögliche Werte des Merkmals, z.B. 80 für Alter
  \item Qualitative Merkmale: Beschreibende Werte zu Messen oder Klassizifieren. Lassen sich nicht zahlenmäßig erfassen. Beispiel: Augenfarbe
  \item Quantitiative Merkmale: Können durch Zahlenwerte beschrieben werden
  \item Skalenniveau: Informationsgehalt innerhalb einer Skala
  \item Permutation: Vertauschen
  \item Quantil: Der Wert, der eine Verteilung in bestimmte Segmente aufteilt.
0.10 Quantil ist der Wert, der die unteren 10\% der Daten von den oberen 90\% trennt
    \item Quartil: Teilt daten in Viertel
   \item Quintil: Fünftelwerte	
\end{itemize}

\subsection{Arten der Statistik}
\subsection{Merkmale und Skalen}

\begin{itemize}

  \item Nominal: Die Werte können nicht geordnet werden, z.B Augenfarbe
  \item Ordinal/Rang: Werke können geordnet werden. Beispiel Größen S, M, L, XL. Mit Ordinalskalen kann nicht gerechnet oder gewertet (z.B ''M ist besser als S'') werden
  \item Metrisch: Ermöglicht das Erfassen von Rängen und Abständen einer Skala
  \begin{itemize}
    \item Intervall: Ermöglicht die Bildung von Differenzen. zB Jahreszahlen. Multiplikation oder Division nicht sinnvoll
    \item Verhältnis: Erlaubt Vergleiche auf Identität, Größe, Additionen, Sutraktionen, Multiplikation und Division. Beispiel Körpergewicht oder Einkommen
  \end{itemize}
\end{itemize}




\subsection{Datenerhebung}